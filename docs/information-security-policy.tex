\documentclass[11pt]{article}
\usepackage{geometry}
\usepackage{enumitem}
\usepackage{hyperref}
\usepackage{titlesec}
\usepackage[utf8]{inputenc}
\geometry{margin=1in}
\titleformat{\section}{\large\bfseries}{\thesection}{0.75em}{}
\setlist[itemize]{leftmargin=1.5em}

\title{Focus Notebook Information Security Policy}
\author{Focus Notebook}
\date{\today}

\begin{document}
\maketitle

\section{Purpose \& Scope}
This policy defines how Focus Notebook protects the confidentiality, integrity, and availability of customer data processed through our Firebase-backed platform. The policy covers all employees, contractors, systems, and vendors that interact with production or customer data, including Google Cloud Platform (GCP), Firebase services, Stripe, Plaid, and AI providers.

\section{Governance}
\begin{itemize}
  \item The Chief Technology Officer (CTO) owns this policy, reviews it at least annually, and reports material risks to company leadership.
  \item All personnel must acknowledge the policy during onboarding and whenever substantive updates are made.
  \item Security requirements are embedded in our SDLC through peer review, automated checks, and deployment gates.
  \item Risks are tracked in the internal issue tracker with owners, mitigation dates, and validation notes.
\end{itemize}

\section{Risk Management}
\begin{itemize}
  \item Conduct a lightweight risk assessment twice per year focused on Firebase, Plaid, Stripe, and data pipeline changes.
  \item Prioritize risks based on impact and likelihood; document resulting controls or compensating measures.
  \item Require architectural review before onboarding new data processors or deploying new Firebase regions.
\end{itemize}

\section{Access Control}
\begin{itemize}
  \item Use Google Workspace identities with enforced MFA for all engineering accounts.
  \item Apply least-privilege IAM roles in GCP/Firebase; production deploy access is limited to the on-call engineering team.
  \item Service accounts are scoped per environment; secrets such as Plaid tokens are injected via Firebase Secret Manager and never committed to source control.
  \item Access reviews occur quarterly; stale accounts or unused roles are removed within five business days.
\end{itemize}

\section{Data Protection}
\begin{itemize}
  \item Data is classified as Public, Internal, Sensitive, or Highly Sensitive. Plaid-derived banking data and PII are Highly Sensitive.
  \item All data in Firebase (Firestore, Storage) inherits Google-managed encryption at rest; TLS 1.2+ is enforced for data in transit.
  \item Sensitive exports (e.g., CSV statements) are deleted once processed unless a product requirement mandates retention; retention schedules are documented per dataset.
  \item Encryption keys are managed by GCP KMS; overrides require CTO approval and change management.
\end{itemize}

\section{Data Retention \& Deletion}
\begin{itemize}
  \item Maintain a living data inventory (Firestore collections, Storage buckets, analytics exports, and logs) with documented lawful bases, retention targets, and disposal procedures aligned with GDPR, CCPA, and other applicable privacy laws.
  \item Highly Sensitive Plaid-derived data is retained only as long as an account remains active plus 24 months for fraud prevention and accounting; manual CSV exports and ad-hoc reports are purged within 30 days unless a shorter contractual requirement applies.
  \item Daily managed backups inherit Google defaults and expire within 35 days; on-demand exports include signed timestamps and destruction dates recorded in the retention register.
  \item Verified data-subject deletion requests are fulfilled within 30 days (or faster when required) by removing data from Firestore, Storage, and analytics tables, then scheduling the next backup rotation to ensure irreversible removal.
  \item Deletion events and retention audits are tracked in the ticketing system with an assigned owner and reviewer; completion is evidenced by automation logs or manual validation notes.
  \item Retention schedules and deletion controls are reviewed at least annually in partnership with legal counsel, and updates are communicated to engineering and support teams.
\end{itemize}

\section{Application \& Infrastructure Security}
\begin{itemize}
  \item Firestore and Storage security rules enforce per-user access, verified via automated tests before deployment.
  \item Cloud Functions use the minimum necessary permissions; environment configuration is stored in Firebase Secrets and Config.
  \item Dependencies are scanned via GitHub Dependabot and npm audit; critical advisories are patched within seven days.
  \item All production changes require pull-request review and automated CI (lint, tests, build) before deployment.
\end{itemize}

\section{Monitoring \& Logging}
\begin{itemize}
  \item Cloud Logging captures application logs, admin actions, and IAM events; logs are retained for at least 90 days.
  \item Alerts notify the on-call engineer of elevated error rates, failed deployments, and suspicious authentication events.
  \item Weekly log reviews confirm no unauthorized access occurred; findings feed into the risk register.
\end{itemize}

\section{Incident Response}
\begin{itemize}
  \item Incidents follow four stages: Detect, Triage, Contain/Eradicate, and Recovery/Postmortem.
  \item Severity definitions determine notification timelines; Plaid and affected customers are notified within 24 hours of confirming exposure of financial data.
  \item Postmortems are completed within five business days and include remediation owners and due dates.
\end{itemize}

\section{Business Continuity \& Backup}
\begin{itemize}
  \item Firestore managed backups or scheduled exports run daily; restore procedures are tested quarterly in a staging project.
  \item Infrastructure-as-code (firebase.json, rules, deployment scripts) is version-controlled to enable rapid rebuilds.
  \item Critical vendor outages (GCP, Plaid, Stripe) are tracked via status feeds; runbooks document fallback steps.
\end{itemize}

\section{Vendor \& Third-Party Management}
\begin{itemize}
  \item Maintain an inventory of critical vendors with contract owner, data processed, and compliance posture (SOC 2/ISO 27001).
  \item Require DPAs and security documentation before onboarding vendors handling Sensitive or Highly Sensitive data.
  \item Review vendor security attestations annually; high-risk vendors undergo penetration testing or independent audits when available.
\end{itemize}

\section{Physical Security \& Device Management}
\begin{itemize}
  \item Production systems run exclusively on Google data centers; we rely on Google’s physical controls (SOC 2 Type II).
  \item Employee devices use full-disk encryption, screen auto-lock (<5 minutes), and company-managed endpoint protection.
  \item Lost or stolen devices are reported within one hour and remotely wiped; replacements require fresh security posture verification.
\end{itemize}

\section{Training \& Awareness}
\begin{itemize}
  \item Security awareness training is required at hire and refreshed annually, covering phishing, data handling, and incident reporting.
  \item Engineers receive role-specific training on Firebase security rules, secret management, and compliance obligations.
\end{itemize}

\section{Policy Maintenance}
\begin{itemize}
  \item This policy is reviewed annually or upon major platform changes; revisions are versioned and communicated to all staff.
  \item Exceptions require written approval from the CTO, include an expiration date, and are tracked in the risk register.
\end{itemize}

\end{document}
